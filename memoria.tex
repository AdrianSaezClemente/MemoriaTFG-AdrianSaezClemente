%%%%%%%%%%%%%%%%%%%%%%%%%%%%%%%%%%%%%%%%%%%%%%%%%%%%%%%%%%%%%%%%%%%%%%%%%%%%%%%%
%% Plantilla de memoria en LaTeX para la ETSIT - Universidad Rey Juan Carlos
%%
%% Por Gregorio Robles <grex arroba gsyc.urjc.es>
%%     Grupo de Sistemas y Comunicaciones
%%     Escuela Tcnica Superior de Ingenieros de Telecomunicacin
%%     Universidad Rey Juan Carlos
%% (muchas ideas tomadas de Internet, colegas del GSyC, antiguos alumnos...
%%  etc. Muchas gracias a todos)
%%
%% La ltima versin de esta plantilla est siempre disponible en:
%%     https://github.com/gregoriorobles/plantilla-memoria
%%
%% Para obtener PDF, ejecuta en la shell:
%%   make
%% (las imgenes deben ir en PNG o JPG)

%%%%%%%%%%%%%%%%%%%%%%%%%%%%%%%%%%%%%%%%%%%%%%%%%%%%%%%%%%%%%%%%%%%%%%%%%%%%%%%%

\documentclass[a4paper, 12pt]{book}
%\usepackage[T1]{fontenc}
\usepackage{hyperref}
\usepackage[a4paper, left=2.5cm, right=2.5cm, top=3cm, bottom=3cm]{geometry}
\usepackage{times}
\usepackage[latin1]{inputenc}
\usepackage[spanish,activeacute]{babel} % Comenta esta lnea si tu memoria es en ingls
\usepackage{url}
%\usepackage[dvipdfm]{graphicx}
\usepackage{fancybox}
\usepackage{subfigure}
\usepackage{graphicx}
\usepackage{float}  %% H para posicionar figuras
\usepackage[nottoc, notlot, notlof, notindex]{tocbibind} %% Opciones de ndice
\usepackage{latexsym}  %% Logo LaTeX

\title{Memoria del Proyecto}
\author{Adri\'an S\'aez Clemente}

\renewcommand{\baselinestretch}{1.5}  %% Interlineado

\begin{document}

%\renewcommand{\refname}{Bibliografa}  %% Renombrando
\renewcommand{\appendixname}{Ap\'endice}

%%%%%%%%%%%%%%%%%%%%%%%%%%%%%%%%%%%%%%%%%%%%%%%%%%%%%%%%%%%%%%%%%%%%%%%%%%%%%%%%
% PORTADA

\begin{titlepage}
\begin{center}
\begin{tabular}[c]{c c}
%\includegraphics[bb=0 0 194 352, scale=0.25]{logo} &
\includegraphics[scale=0.25]{img/logo_vect.png} &
\begin{tabular}[b]{l}
\Huge
\textsf{UNIVERSIDAD} \\
\Huge
\textsf{REY JUAN CARLOS} \\
\end{tabular}
\\
\end{tabular}

\vspace{3cm}

\Large
GRADO EN INGENIER\'IA EN TECNOLOG\'IAS DE LA TELECOMUNICACI\'ON

\vspace{0.4cm}

\large
Curso Acad\'emico 2014/2015

\vspace{0.8cm}

Trabajo Fin de Grado

\vspace{2.5cm}

\LARGE
DESARROLLO Y DESPLIEGUE DE APLICACIONES EN EL \'AMBITO EDUCATIVO

\vspace{4cm}

\large
Autor : Adri\'an S\'aez Clemente \\
Tutor : Gregorio Robles Mart\'inez
\end{center}
\end{titlepage}

\newpage
\mbox{}
\thispagestyle{empty} % para que no se numere esta pagina


%%%%%%%%%%%%%%%%%%%%%%%%%%%%%%%%%%%%%%%%%%%%%%%%%%%%%%%%%%%%%%%%%%%%%%%%%%%%%%%%
%%%% Dedicatoria

%\chapter*{}
%\pagenumbering{Roman} % para comenzar la numeracion de paginas en numeros romanos
%\begin{flushright}
%\textit{Dedicado a \\
%mi familia / mi abuelo / mi abuela}
%\end{flushright}

%%%%%%%%%%%%%%%%%%%%%%%%%%%%%%%%%%%%%%%%%%%%%%%%%%%%%%%%%%%%%%%%%%%%%%%%%%%%%%%%
%%%% Agradecimientos

\chapter*{Agradecimientos}
\addcontentsline{toc}{chapter}{Agradecimientos} % si queremos que aparezca en el ndice
\markboth{AGRADECIMIENTOS}{AGRADECIMIENTOS} % encabezado 

Aqu vienen los agradecimientos\ldots Aunque est bien acordarse de la pareja,
no hay que olvidarse de dar las gracias a tu madre, que aunque a veces no lo 
parezca disfrutar tanto de tus logros como t\ldots Adems, la pareja quizs
no sea para siempre, pero tu madre s.

%%%%%%%%%%%%%%%%%%%%%%%%%%%%%%%%%%%%%%%%%%%%%%%%%%%%%%%%%%%%%%%%%%%%%%%%%%%%%%%%
%%%% Resumen

\chapter*{Resumen}
\addcontentsline{toc}{chapter}{Resumen} % si queremos que aparezca en el ndice
\markboth{RESUMEN}{RESUMEN} % encabezado

En este proyecto se realiza el desarrollo de una aplicaci\'on web que se basa en la creaci\'on de una plataforma....

\begin{itemize}
  \item De qu va este proyecto? Cul es su objetivo principal?
  \item Cmo se ha realizado? Qu tecnologas estn involucradas?
  \item En qu contexto se ha realizado el proyecto? Es un proyecto
dentro de un marco general?
\end{itemize}

Lo mejor es escribir el resumen al final.

%%%%%%%%%%%%%%%%%%%%%%%%%%%%%%%%%%%%%%%%%%%%%%%%%%%%%%%%%%%%%%%%%%%%%%%%%%%%%%%%
%%%% Resumen en ingls

\chapter*{Summary}
\addcontentsline{toc}{chapter}{Summary} % si queremos que aparezca en el ndice
\markboth{SUMMARY}{SUMMARY} % encabezado

Here comes a translation of the ``Resumen'' into English. Please, double check
it for correct grammar and spelling. As it is the translation of the ``Resumen'',
which is supposed to be written at the end, this as well should be filled out
just before submitting.


%%%%%%%%%%%%%%%%%%%%%%%%%%%%%%%%%%%%%%%%%%%%%%%%%%%%%%%%%%%%%%%%%%%%%%%%%%%%%%%%
%%%%%%%%%%%%%%%%%%%%%%%%%%%%%%%%%%%%%%%%%%%%%%%%%%%%%%%%%%%%%%%%%%%%%%%%%%%%%%%%
% NDICES %
%%%%%%%%%%%%%%%%%%%%%%%%%%%%%%%%%%%%%%%%%%%%%%%%%%%%%%%%%%%%%%%%%%%%%%%%%%%%%%%%

% Las buenas noticias es que los ndices se generan automticamente.
% Lo nico que tienes que hacer es elegir cules quieren que se generen,
% y comentar/descomentar esa instruccin de LaTeX.

%%%% ndice de contenidos
\tableofcontents 
%%%% ndice de figuras
\cleardoublepage
\phantomsection
\addcontentsline{toc}{chapter}{Lista de figuras} % para que aparezca en el indice de contenidos
\listoffigures % indice de figuras
%%%% ndice de tablas
%\cleardoublepage
%\addcontentsline{toc}{chapter}{Lista de tablas} % para que aparezca en el indice de contenidos
%\listoftables % indice de tablas


%%%%%%%%%%%%%%%%%%%%%%%%%%%%%%%%%%%%%%%%%%%%%%%%%%%%%%%%%%%%%%%%%%%%%%%%%%%%%%%%
%%%%%%%%%%%%%%%%%%%%%%%%%%%%%%%%%%%%%%%%%%%%%%%%%%%%%%%%%%%%%%%%%%%%%%%%%%%%%%%%
% INTRODUCCIN %
%%%%%%%%%%%%%%%%%%%%%%%%%%%%%%%%%%%%%%%%%%%%%%%%%%%%%%%%%%%%%%%%%%%%%%%%%%%%%%%%

\cleardoublepage
\chapter{Introducci\'on}
\label{sec:intro} % etiqueta para poder referenciar luego en el texto con ~\ref{sec:intro}
\pagenumbering{arabic} % para empezar la numeracin de pgina con nmeros

\section{Descripci\'on del proyecto}
\label{sec:descripcion}
En este proyecto se implementa una aplicaci\'on web que consiste en el desarrollo de un planeta de blogs. Est\'a enfocado al \'ambito educativo y puede
ser usado en cualquier entorno que simule un aula, ya sea en universidades, institutos, colegios, academias, cursos online, etc,...

El contenido de la aplicaci\'on recoge autom\'aticamente mensajes o entradas de un conjunto de blogs que comparten una tem\'atica com\'un.

El tutor ser\'a el responsable de crear hilos donde los alumnos podr\'an suscribirse y formar parte de la tem\'atica de inter\'es. Dentro de cada hilo,
los alumnos podr\'an visualizar las entradas del conjunto de blogs que est\'an suscritos a ese hilo.

Como novedad incluye la posibilidad de interactuar, valorar y comentar los mensajes de compa\~neros de una forma divertida mediante un proceso de 
gamificaci\'on. En este proceso los participantes de cada hilo podr\'an luchar por un puesto en la clasificaci\'on definitiva consiguiendo 
puntos (llamados planets) y subiendo de nivel seg\'un el compromiso y la capacidad de interacci\'on de cada alumno.

Existe una secci\'on de b\'usqueda para poder encontrar entradas introduciendo nick de usuario y/o identificador de entrada. Y adem\'as, dentro de la
aplicaci\'on se puede encontrar otra secci\'on con las bases de las puntuaciones.

\section{Contexto tecnol\'ogico}
\label{sec:contexto}








%%%%%%%%%%%%%%%%%%%%%%%%%%%%%%%%%%%%%%%%%%%%%%%%%%%%%%%%%%%%%%%%%%%%%%%%%%%%%%%%
%%%%%%%%%%%%%%%%%%%%%%%%%%%%%%%%%%%%%%%%%%%%%%%%%%%%%%%%%%%%%%%%%%%%%%%%%%%%%%%%
% OBJETIVOS %
%%%%%%%%%%%%%%%%%%%%%%%%%%%%%%%%%%%%%%%%%%%%%%%%%%%%%%%%%%%%%%%%%%%%%%%%%%%%%%%%

\cleardoublepage
\chapter{Objetivos}
\label{chap:objetivos}

\section{Objetivo general}
label{sec:objetivo-general}


\section{Objetivos especficos}
label{sec:objetivos-especificos}


\section{Planificacin temporal}
label{sec:planificacion-temporal}



%%%%%%%%%%%%%%%%%%%%%%%%%%%%%%%%%%%%%%%%%%%%%%%%%%%%%%%%%%%%%%%%%%%%%%%%%%%%%%%%
%%%%%%%%%%%%%%%%%%%%%%%%%%%%%%%%%%%%%%%%%%%%%%%%%%%%%%%%%%%%%%%%%%%%%%%%%%%%%%%%
% ESTADO DEL ARTE %
%%%%%%%%%%%%%%%%%%%%%%%%%%%%%%%%%%%%%%%%%%%%%%%%%%%%%%%%%%%%%%%%%%%%%%%%%%%%%%%%

\cleardoublepage
\chapter{Estado del arte}

Descripcin de las tecnologas que utilizas en tu trabajo. Con dos o tres prrafos por cada tecnologa, vale.


Puedes citar libros, como el de Bonabeau et al. sobre procesos estigmrgicos~\cite{bonabeau:_swarm}. % Nota que el ~ aade un espacio en blanco, pero no deja que exista un salto de lnea. Imprescindible ponerlo para las citas.

Tambin existe la posibilidad de poner notas al pie de pgina, por ejemplo, 
una para indicarte que visite la pgina de 
LibreSoft\footnote{\url{http://www.libresoft.es}}.

\section{Django y Python} 
\label{sec:djangopython}

\subsection{Django} 
\label{subsec:django}
Django es un framework de desarrollo web de c\'odigo abierto. Est\'a escrito en Python, y respeta el paradigma Modelo-Vista-Controlador. 
Este patr\'on se usa para desarrollar software y se basa en la separaci\'on de los datos, de la interfaz del usuario y de la l\'ogica interna. 
En aplicaciones web se usa con mucha frecuencia, d\'onde la vista es la p\'agina HTML, el modelo es el Sistema de Gesti\'on de Base de Datos y la 
l\'ogica interna, y el controlador es el responsable de recibir los eventos y darles soluci\'on.

-Modelo: Es la capa de acceso a la base de datos. Esta capa contiene toda la informaci\'on sobre los datos: como acceder a estos, como validarlos, 
cual es el comportamiento que tiene, y las relaciones entre los datos. Se utiliza como representaci\'on de la informaci\'on en el sistema. Trabaja junto 
a la vista para mostrar la informaci\'on al usuario y es accedido por el controlador para a\~nadir, eliminar, consultar o actualizar datos. 
     
-Vista: Es la capa de la l\'ogica de negocios. Esta capa contiene la l\'ogica que accede al modelo y la delega a la plantilla apropiada. Se presenta 
como un puente entre los modelos y las plantillas.

-Controlador: Es el elemento m\'as abstracto. Recibe, trata y responde los eventos enviados por el usuario o por la propia aplicaci\'on. Interactua 
tanto con el modelo como con la vista.



\subsection{Python} 
\label{subsec:python}
Python es un lenguaje de programaci\'on con una sintaxis muy limpia y que favorece un c\'odigo legible. Se ejecuta utilizando un programa intermedio 
llamado int\'erprete, en lugar de compilar el c\'odigo a lenguaje m\'aquina que pueda comprender
y ejecutar directamente una computadora. Tiene una ejecuci\'on m\'as lenta que los lenguajes compilados pero es m\'as flexible y portable.

Tiene la caracter\'istica de tipado din\'amico que se refiere a que no es necesario declarar el tipo de dato que va a contener una determinada 
variable, sino que su tipo se determinar\'a en tiempo de ejecuci\'on seg\'un el tipo del valor al que se asigne, y el tipo de esta variable puede 
cambiar si se le asigna un valor de otro tipo. Adem\'as, Python es un lenguaje fuertemente tipado que no permite tratar a una variable como si fuera 
de un tipo distinto al que tiene, es necesario convertir de forma expl\'icita dicha variable al nuevo tipo previamente. 

El int\'erprete de Python est\'a disponible en multitud de plataformas (UNIX, Solaris, Linux, DOS, Windows, OS/2, Mac OS, etc,..) 
por lo que si no utilizamos librer\'ias espec\'ificas de cada plataforma nuestra aplicaci\'on podr\'a correr en todos estos sistemas sin grandes 
cambios.

Es un lenguaje orientado a objetos. Este concepto se define como un paradigma de programaci\'on en el que los conceptos del mundo real relevantes 
para nuestro problema se trasladan a clases y objetos en nuestra aplicaci\'on. La ejecuci\'on de la aplicaci\'on consiste en una serie de interacciones 
entre los objetos.

En este proyecto, Python se usa principalmente para el desarrollo de las vistas en el lado del servidor, donde mediante 
peticiones de los usuarios, recoge y procesa la informaci\'on y devuelve los datos a la presentaci\'on correspondiente.


\section{HTML5}
\label{sec:html5}
HTML5 (HyperText Markup Language) Es la versi\'on 5 del lenguaje en el que se escriben las p\'aginas webs. Un lenguaje que s\'olo son capaces 
de leer los \'ultimos navegadores, ya que los que son m\'as anticuados hay algunas etiquetas que no son capaces de interpretar. Sirve para integrar 
contenidos multimedia, flash y para dar un sentido sem\'antico a cada parte de la p\'agina web.

HTML5 disminuye el tiempo de carga de las p\'aginas y tambi\'en ayuda a bajar el ratio de texto/HTML ya que es un lenguaje m\'as simple.
Es un lenguaje m\'as sem\'antico que dota de un significado a las diferentes partes de la p\'agina Web. Ayuda a los buscadores a entender la p\'agina 
web y tambi\'en sirve para que los navegadores muestren de forma inteligente diferentes versiones de la p\'agina Web sin tener que usar plug-ins para 
dispositivos espec\'ificos.

HTML5 incluye m\'as elementos gr\'aficos y multimedia. Permite la inserci\'on de etiquetas canvas, que sustituyen a las animaciones en Flash y tambi\'en 
permite incluir de forma muy sencilla v\'ideos, m\'usica y otros elementos multimedia como por ejemplo los videojuegos. Adem\'as incluye la 
geolocalizaci\'on. Es posible saber desde que lugar se esta visualizando un sitio Web, gracias a sistemas de referenciaci\'on como el GPS, 
la tecnolog\'ia 3G de los dispositivos m\'oviles o las conexiones tipo WiFi.

HTML5 permite el uso de las Webs offline. Aunque no tengas conexi\'on a Internet si el programador lo desea podr\'as ver parte o toda la p\'agina Web, 
e incluso interactuar con los contenidos. Esto mejora la usabilidad de la Web y permite al usuario trabajar con aplicaciones online desconectado, 
algo importante en los dispositivos m\'oviles cuando te mueves en entornos en los que no tienes una conexi\'on de calidad.


\section{JavaScript}
\label{sec:javascript}


\subsection{jQuery}
\label{subsec:jquery}
jQuery es una librer\'ia JavaScript pensada y creada para acceder f\'acilmente a los elementos del DOM, para automatizar tareas comunes 
y simplificar las labores complicadas de JavaScript.

Esta librer\'ia da una serie de ventajas: 
\begin{itemize}
  \item Facilidad de acceso en el documento.
  \item Para la localizaci\'on de partes espec\'ificas en la estructura del documento HTML, jQuery tiene un mecanismo de selector s\'olido y 
eficiente con el que puede recuperar piezas exactas del DOM (Document Object Model).
  \item Modifica la apariencia de un sitio web. Puede llenar vacios que CSS no puede cubrir. Y tambi\'en puede cambiar clases y estilos aplicados 
en CSS incluso despu\'es de que la p\'agina ya est\'e renderizada en el navegador.
  \item Altera contenidos del documento. JQuery puede modificar cualquier contenido en el documento.
  \item Responde a la interacci\'on del usuario. Ofrece una forma elegante de controlar acciones y nos presenta una amplia variedad de eventos 
que responden a diversas acciones del usuario sin necesidad de saturar el c\'odigo HTML.
  \item Simplifica tareas comunes de JavaScript. Esta librer\'ia ofrece mejoras en el momento de desarrollar en JavaScript, nos ahorra tiempo 
y el c\'odigo ser\'a siempre m\'as limpio y sencillo.
\end{itemize}


\subsection{jQuery-ui}
\label{subsec:jqueryui}
jQuery-ui es una biblioteca de componentes para la librer\'ia jQuery que le a\~naden un conjunto de plug-ins, widgets y efectos visuales para la 
creaci\'on de aplicaciones web.


\subsection{AJAX}
\label{subsec:ajax}
AJAX (Asynchronous JavaScript And XML) se usa para la transferencia de informaci\'on mediante JavaScript. Esta informaci\'on fluye por dos canales 
entre el cliente y el servidor, y puede ser transferida en formato XML, texto simple u otros est\'andares como JSON.

AJAX sirve para actualizar una porci\'on de nuestra p\'agina web sin necesidad de cargar el c\'odigo, atributos gr\'aficos e im\'agenes de nuevo. Esto
provoca mayor rapidez en la interacci\'on cliente-servidor. El cliente no percibe demoras puesto que las comunicaciones se producen en segundo plano y
no hay interrupciones. AJAX tiene la capacidad de comprimir las funciones de una aplicaci\'on en un espacio HTML m\'as reducido de lo que antes pod\'ia
ser necesario.

Algunos de los inconvenientes de AJAX son el aumento de la complejidad en el desarrollo de aplicaciones, problemas con restricciones de seguridad 
relacionados con su uso, y aumento de la dificultad en la indexaci\'on para los motores de b\'usqueda ya que el contenido deja de ser est\'atico.


\subsection{JSON}
\label{subsec:json}
JSON es un formato de fichero de texto ligero utilizado para el intercambios de datos. B\'asicamente JSON describe los datos con una sintaxis dedicada 
que se usa para identificar y gestionar los datos. Este formato es simple, directo, ligero y muy importante para el desarrollo de aplicaciones basadas 
en lenguajes Web y para dispositivos m\'oviles. Cada dato es un par nombre/valor.


\section{CSS3}
\label{sec:css3}
CSS (Cascading Style Sheets) se refiere a la tecnolog\'ia desarrollada para separar la presentaci\'on de la estructura HTML de un sitio web. 
Esta tecnolog\'ia aplica reglas de estilo a los elementos HTML, quedando de esta manera separada de la estructura HTML. Poco a poco este lenguaje se 
ha ido haciendo m\'as importante entre los dise\~nadores gracias a toda la facilidad de uso, y los resultados que son muy flexibles.

CSS3 es la versi\'on 3 de CSS, donde se definen las caracter\'isticas de este lenguaje, a\~nadiendo muchas funciones y mejorado otras.
En una hoja de estilo \'o fichero .css, se definen reglas en las cuales se van a aplicar diferentes y muy variados efectos a las etiquetas HTML.
Las reglas se forman a partir de selectores, atributos y valores. 

Los selectores se usan para definir sobre qu\'e etiqueta HTML vamos a aplicar atributos y valores. Existen selectores de etiquetas HTML, los cuales 
se utilizan escribiendo el nombre de la etiqueta a la que se aplicar\'a el estilo. Tambi\'en existen selectores de identificador, que son aquellos que 
empiezan por ``\#'' seguido del id de la etiqueta a la que queremos darle los atributos. Y por \'ultimo, selectores de clase que se escriben en el 
documento CSS comenzando por ``.'' y va seguido del nombre de la clase de la etiqueta.

Los atributos son efectos o estilos que aplicaremos a las etiquetas HTML. Se aplican sobre texto, enlaces, im\'agenes, listas, tablas, formularios, 
etc,.. Tambi\'en incluye la posibilidad de realizar animaciones con porcentajes de tiempo.

Una vez definidos los selectores y los atributos, se les da un valor a los atributos. 


\section{BootStrap}
\label{sec:bootstrap}
Bootstrap es un framework originalmente creado por Twitter, que permite crear interfaces web con CSS y JavaScript y cuya particularidad es la de adaptar 
la interfaz del sitio web al tama\~no del dispositivo en que se visualice, ya sea pantalla de PC, port\'atil, tablet, Smartphone,.. 
Esta t\'ecnica de dise\~no y desarrollo se conoce como dise\~no adaptativo o `responsive design'.
El beneficio de usar dise\'no adaptativo en un sitio web, es principalmente que el sitio web se adapta autom\'aticamente al dispositivo desde donde 
se acceda. Aun ofreciendo todas las posibilidades que ofrece Bootstrap a la hora de crear interfaces web, los dise\'nos creados con Bootstrap son 
simples, limpios e intuitivos, esto les da agilidad a la hora de cargar y adaptarse a otros dispositivos. El framework trae varios elementos con 
estilos predefinidos f\'aciles de configurar como botones, men\'us desplegables, formularios incluyendo todos sus elementos, y jQuery integrado para 
ofrecer ventanas y tooltips din\'amicos.
El uso de Bootstrap ofrece una serie de ventajas:
\begin{itemize}
  \item Permite tener un mismo lenguaje para la comunicaci\'on entre dise\~nadores y equipo de desarrollo.
  \item Tiene un comportamiento unificado entre diferentes navegadores.
  \item Es compatible con navegadores modernos, dispositivos y navegadores antiguos (hasta Internet Explorer 8)
  \item Soporte avanzado para Responsive web Design (se podr\'ia utilizar como NO responsive pero forz\'andolo).
  \item Comportamiento y plugins basados en jQuery muy \'utiles y usados habitualmente.
  \item Bootstrap es responsive y viene con un grid de 12 columnas y cuenta con diferentes `Media queries': 768 px (tablets), 992 px (Desktop) y 
  1200 px (Large Desktop) + Mobile (por defecto en Bootstrap).
  \item Y por \'ultimo y m\'as importante, usa un sistema de rejillas.
\end{itemize}

Bootstrap incluye una rejilla flu\'ida pensada para m\'oviles. Esta rejilla crece hasta 12 columnas a medida que crece el tama\~no de la pantalla del 
dispositivo. Bootstrap incluye clases CSS para utilizar la rejilla directamente en tus dise\~nos y tambi\'en define mixins de LESS para que puedas 
crear dise\~nos m\'as sem\'anticos.

\begin{figure}[htbp] 
  \label{figura:bootstrap}
  \centering
  \shadowbox{\includegraphics[width=17cm, keepaspectratio]{img/BootStrap}}
  \caption{Tama\~nos en diferentes dispositivos}
\end{figure}

El dise\~no de p\'aginas basado en rejilla se realiza mediante filas y columnas donde se colocan los contenidos. La rejilla de Bootstrap funciona del
siguiente modo: Las filas siempre se definen dentro de un contenedor de clase .container (anchura fija) o de tipo .container-fluid (anchura variable), 
de esta forma las filas se alinean bien y muestran el padding correcto. Las filas se utilizan para agrupar horizontalmente a varias columnas. 
El contenido siempre se coloca dentro de las columnas, ya que las filas s\'olo deber\'ian contener como hijos elementos de tipo columna. Bootstrap 
define muchas clases CSS (como por ejemplo .row y .col-xs-4) para crear rejillas r\'apidamente. La separaci\'on entre columnas se realiza aplicando 
padding. Para contrarrestar sus efectos en la primera y \'ultima columnas, las filas (elementos .row) aplican m\'argenes negativos. Las columnas de la 
rejilla definen su anchura especificando cu\'antas de las 12 columnas de la fila ocupan. Si por ejemplo quieres dividir una fila en tres columnas 
iguales, utilizar\'ias la clase .col-xs-4 (el 4 indica que cada columna ocupa 4 de las 12 columnas en las que se divide cada fila).

\begin{figure}[htbp] 
  \label{figura:bootstrap1}
  \centering
  \subfigure[Dispositivo mediano]{\includegraphics[width=11cm]{img/BootStrapMD}}
  \subfigure[Dispositivo peque\~no]{\includegraphics[width=11cm]{img/BootStrapXS}}
  \caption{Ejemplo para dispositivos medianos (clase col-md-x) y peque\~nos (clase col-xs-x)}
\end{figure}

Por \'ultimo, hay que decir que Bootstrap permite el anidamiento, el reordenamiento y el desplazamiento de las columnas 
(tanto por la izquierda como por la derecha).

\section{Git}
\label{sec:git}
Git es un sistema de control de versiones distribuido de c\'odigo abierto relativamente nuevo que nos ofrece las mejores caracter\'isticas en la 
actualidad, pero sin perder la sencillez. Es el sistema de control de versiones m\'as usado por desarrolladores en el mundo. 
A los programadores ayuda a ser m\'as eficientes en su trabajo, ya que ha universalizado las herramientas de control de versiones del software que 
hasta entonces no estaban tan popularizadas y tan al alcance del com\'un de los desarrolladores.

Git es multiplataforma, por lo que puedes usarlo y crear repositorios locales en todos los sistemas operativos m\'as comunes, Windows, Linux o Mac. 
Existen multitud de GUIs (Graphical User Interface o Interfaz de Usuario Gr\'afica) para trabajar con Git. Una de ellas es por l\'inea de comandos.

Dentro de este proyecto, para hacer uso de este sistema de control de versiones se ha utilizado Github. Github es un servicio para alojamiento de 
repositorios de software gestionados por el sistema de control de versiones Git. Por tanto, Git es algo m\'as general que nos sirve para controlar 
el estado de un desarrollo a lo largo del tiempo, mientras que Github es un sitio web que usa Git para ofrecer a la comunidad de desarrolladores 
repositorios de software. En definitiva, Github es un sitio web pensado para hacer posible el compartir el c\'odigo de una manera m\'as f\'acil.


\section{Pivotal tracker}
\label{sec:pivotal}
Pivotal Tracker es un servicio para la gesti\'on \'agil de proyectos y colaboraci\'on creado por Pivotal Labs. Esta herramienta incluye soporte 
para el intercambio de archivos y gesti\'on de tareas, as\'i como para la planificaci\'on de las iteraciones. Adem\'as posee una API para extensiones y 
herramientas de terceros.


\section{Crontab}
\label{sec:crontab}
https://www.open6hosting.com/crontab/


%%%%%%%%%%%%%%%%%%%%%%%%%%%%%%%%%%%%%%%%%%%%%%%%%%%%%%%%%%%%%%%%%%%%%%%%%%%%%%%%
%%%%%%%%%%%%%%%%%%%%%%%%%%%%%%%%%%%%%%%%%%%%%%%%%%%%%%%%%%%%%%%%%%%%%%%%%%%%%%%%
% DISEO E IMPLEMENTACIN %
%%%%%%%%%%%%%%%%%%%%%%%%%%%%%%%%%%%%%%%%%%%%%%%%%%%%%%%%%%%%%%%%%%%%%%%%%%%%%%%%

\cleardoublepage
\chapter{Dise\~no e implementaci\'on}

\section{Arquitectura general} 
\label{sec:arquitectura}
En esta secci\'on se define la estructura interna del proyecto. En la figura~\ref{fig:arquitectura} se muestra una imagen detallada de los directorios y 
subdirectorios con sus respectivos archivos que hacen posible el funcionamiento de la aplicaci\'on.
En dicha estructura cabe destacar los siguientes bloques:
\begin{itemize}
  \item {\bfseries actualizar.py}: Es un script Python que realiza la funci\'on de actualizaci\'on autom\'atica de entradas. Cada vez que un alumno ingresa
  una nueva entrada en su blog, este script es el encargado de detectar esos nuevos cambios a trav\'es de los RSS guardados en base de datos.
  \item {\bfseries bbdd.sqlite}: Es la base de datos del proyecto donde se van a guardar todos los datos estructurados a partir de los modelos definidos en 
  \texttt{models.py}
  \item {\bfseries planetablogs}: Contiene la parte fundamental de la funcionalidad del proyecto. Dentro de este directorio se encuentran distintos ficheros
  python. \texttt{admin.py} hace la funci\'on de modificar la interfaz de administraci\'on que \textit{Django} genera por defecto. En \texttt{formularios.py}
  se definen todos los formularios que ser\'an utilizados por los usuarios dentro de la aplicaci\'on para crear nuevos elementos o actualizar elementos
  existentes en la base de datos. En \texttt{models.py} se encuentran los modelos que se convertir\'an en tablas dentro de la base de datos. Estos modelos se
  encargan de representar los datos dentro de la aplicaci\'on y contienen los campos necesarios y el comportamiento de los datos que ser\'an almacenados.
  Por \'ultimo, \texttt{views.py} es un fichero donde se implementa toda la l\'ogica del proyecto llevando a cabo todas las peticiones de los usuarios.
  \item {\bfseries plantillas}: En este directorio se encuentran todas las plantillas de los documentos \textit{HTML} que ser\'an enviados al usuario y 
  mostrados por el navegador para dar una interfaz gr\'afica a la aplicaci\'on.
  \item {\bfseries static}: Esta carpeta contiene las librer\'ias \textit{BootStrap}, \textit{jQuery} y \textit{jQuery-ui}. Tambi\'en incluye un subdirectorios
  donde se guardan tanto las im\'agenes propias del proyecto como las im\'agenes de perfil introducidas por los usuarios como informaci\'on de registro. 
  Y adem\'as, se localizan todos los ficheros \textit{.js} y \textit{.css} que a\~nadir\'an funcionalidad y definir\'an el dise\~no a su respectivo 
  documento \textit{HTML} o p\'agina web.
  \item {\bfseries TFG}: Contiene ficheros b\'asicos para el correcto funcionamiento de la aplicaci\'on.\\ \texttt{settings.py} es un fichero de 
  configuraci\'on que contiene todas las rutas relativas o directorios necesarios. \texttt{urls.py} define todas las urls y las asocia a una vista en
  \texttt{views.py} que se encargar\'a de procesar las peticiones de los clientes.
\end{itemize}


\begin{figure}
  \centering
  \includegraphics[width=9cm, keepaspectratio]{img/tree}
  \caption{Estructura de directorios. Utilizando la herramienta \textit{tree}.}
  \label{fig:arquitectura}
\end{figure}


\section{Estructura \textit{Django} del proyecto} 
\label{sec:estructuradjango}
En esta secci\'on se expone una descripci\'on m\'as detallada del funcionamiento del proyecto. Se empieza con una introducci\'on que cuenta la estructura
o esqueleto de \textit{Django} y termina con una explicaci\'on m\'as espec\'ifica del dise\~no de las distintas partes implementadas en \'el.


\subsection{Introducci\'on a la estructura de \textit{Django}} 
\label{sec:introestructura}

\begin{figure}
  \centering
  \includegraphics[width=17cm, keepaspectratio]{img/diagrama}
  \caption{\textit{Estructura general \textit{Django}.}}
  \label{fig:introestructura}
\end{figure}


\subsection{Dise\~no de la base de datos} 
\label{sec:disenobbdd}

\begin{figure}
  \centering
  \includegraphics[width=17cm, keepaspectratio]{img/mis_modelos}
  \caption{\textit{Estructura gr\'afica de la base de datos.}}
  \label{fig:disenobbdd}
\end{figure}


\subsection{Dise\~no de las vistas y urls} 
\label{sec:vistasurls}


%%%%%%%%%%%%%%%%%%%%%%%%%%%%%%%%%%%%%%%%%%%%%%%%%%%%%%%%%%%%%%%%%%%%%%%%%%%%%%%%
%%%%%%%%%%%%%%%%%%%%%%%%%%%%%%%%%%%%%%%%%%%%%%%%%%%%%%%%%%%%%%%%%%%%%%%%%%%%%%%%
% RESULTADOS %
%%%%%%%%%%%%%%%%%%%%%%%%%%%%%%%%%%%%%%%%%%%%%%%%%%%%%%%%%%%%%%%%%%%%%%%%%%%%%%%%

\cleardoublepage
\chapter{Resultados}




%%%%%%%%%%%%%%%%%%%%%%%%%%%%%%%%%%%%%%%%%%%%%%%%%%%%%%%%%%%%%%%%%%%%%%%%%%%%%%%%
%%%%%%%%%%%%%%%%%%%%%%%%%%%%%%%%%%%%%%%%%%%%%%%%%%%%%%%%%%%%%%%%%%%%%%%%%%%%%%%%
% CONCLUSIONES %
%%%%%%%%%%%%%%%%%%%%%%%%%%%%%%%%%%%%%%%%%%%%%%%%%%%%%%%%%%%%%%%%%%%%%%%%%%%%%%%%

\cleardoublepage
\chapter{Conclusiones}
\label{chap:conclusiones}


\section{Consecucin de objetivos}
\label{sec:consecucion-objetivos}

Esta seccin es la seccin espejo de las dos primeras del captulo de objetivos,
donde se planteaba el objetivo general y se elaboraban los especficos.

Es aqu donde hay que debatir qu se ha conseguido y qu no. Cuando algo no
se ha conseguido, se ha de justificar, en trminos de qu problemas se han
encontrado y qu medidas se han tomado para mitigar esos problemas.


\section{Aplicacin de lo aprendido}
\label{sec:aplicacion}

Aqu viene lo que has aprendido durante el Grado/Mster y que has aplicado
en el TFG/TFM. Una buena idea es poner las asignaturas ms relacionadas y
comentar en un prrafo los conocimientos y habilidades puestos en prctica.

\begin{enumerate}
  \item a
  \item b
\end{enumerate}


\section{Lecciones aprendidas}
\label{sec:lecciones_aprendidas}

Aqu viene lo que has aprendido en el Trabajo Fin de Grado/Mster.

\begin{enumerate}
  \item a
  \item b
\end{enumerate}


\section{Trabajos futuros}
\label{sec:trabajos_futuros}

Ningn software se termina, as que aqu vienen ideas y funcionalidades
que estara bien tener implementadas en el futuro.

Es un apartado que sirve para dar ideas de cara a futuros TFGs/TFMs.


\section{Valoracin personal}
\label{sec:valoracion}

Finalmente (y de manera opcional), hay gente que se anima a dar su punto de
vista sobre el proyecto, lo que ha aprendido, lo que le gustara haber aprendido,
las tecnologas utilizadas y dems.



%%%%%%%%%%%%%%%%%%%%%%%%%%%%%%%%%%%%%%%%%%%%%%%%%%%%%%%%%%%%%%%%%%%%%%%%%%%%%%%%
%%%%%%%%%%%%%%%%%%%%%%%%%%%%%%%%%%%%%%%%%%%%%%%%%%%%%%%%%%%%%%%%%%%%%%%%%%%%%%%%
% APNDICE(S) %
%%%%%%%%%%%%%%%%%%%%%%%%%%%%%%%%%%%%%%%%%%%%%%%%%%%%%%%%%%%%%%%%%%%%%%%%%%%%%%%%

\cleardoublepage
\appendix
\chapter{Manual de usuario}
\label{app:manual}

\section{P\'agina de inicio}
\begin{figure}[htbp] 
  \label{figura:inicio}
  \centering
  \shadowbox{\includegraphics[width=17cm, keepaspectratio]{imagenes/PaginaInicio}}
  \caption{P\'agina de inicio}
\end{figure}
\begin{figure}[htbp] 
  \label{figura:inicio1}
  \centering
  \subfigure[Inicio de sesi\'on]{\includegraphics[width=6cm]{imagenes/InicioComoAlumno}}
  \subfigure[Inicio de sesi\'on err\'onea]{\includegraphics[width=6cm]{imagenes/InicioError}}
  \caption{Inicio de sesi\'on}
\end{figure}
\newpage
\paragraph{}
Como se muestra en la figura


\newpage
\section{P\'agina de registro}
\begin{figure}[htbp] 
  \label{figura:registro}
  \centering
  \shadowbox{\includegraphics[width=17cm, keepaspectratio]{imagenes/RegistroAlumno}}
  \caption{Registro como alumno}
\end{figure}
\begin{figure}[htbp] 
  \label{figura:registro1}
  \centering
  \shadowbox{\includegraphics[width=17cm, keepaspectratio]{imagenes/RegistroTutor}}
  \caption{Registro como tutor}
\end{figure}
\begin{figure}[htbp] 
  \label{figura:registro2}
  \centering
  \shadowbox{\includegraphics[width=17cm, keepaspectratio]{imagenes/NickCorrectoOcupado}}
  \caption{Rellenar nick}
\end{figure}
\begin{figure}[htbp] 
  \label{figura:registro3}
  \centering
  \shadowbox{\includegraphics[width=17cm, keepaspectratio]{imagenes/RegistroAlumnoError}}
  \caption{Registro err\'oneo}
\end{figure}
\begin{figure}[htbp] 
  \label{figura:registro4}
  \centering
  \shadowbox{\includegraphics[width=17cm, keepaspectratio]{imagenes/RegistroAlumnoInfo}}
  \caption{Informaci\'on de registro}
\end{figure}
se muestra todo

\newpage
\section{Presentaci\'on de tutor}
\begin{figure}[htbp] 
  \label{figura:tutor}
  \centering
  \shadowbox{\includegraphics[width=17cm, keepaspectratio]{imagenes/PresentacionTutor}}
  \caption{P\'agina de presentaci\'on de tutor}
\end{figure}
figura \hyperref[figura:registro1]{A.4}
\newpage
%%%%%%%%%%%%%%%%%%%%%%%%%%%%%%%%%%%%%%%%%%%%%%%%%%%%%%%%%%%%%%%%%%%%%%%%%%%%%%%%
%%%%%%%%%%%%%%%%%%%%%%%%%%%%%%%%%%%%%%%%%%%%%%%%%%%%%%%%%%%%%%%%%%%%%%%%%%%%%%%%
% BIBLIOGRAFIA %
%%%%%%%%%%%%%%%%%%%%%%%%%%%%%%%%%%%%%%%%%%%%%%%%%%%%%%%%%%%%%%%%%%%%%%%%%%%%%%%%

\cleardoublepage

% Las siguientes dos instrucciones es todo lo que necesitas
% para incluir las citas en la memoria
\bibliographystyle{abbrv}
\bibliography{memoria}  % memoria.bib es el nombre del fichero que contiene
% las referencias bibliogrficas. Abre ese fichero y mira el formato que tiene,
% que se conoce como BibTeX. Hay muchos sitios que exportan referencias en
% formato BibTeX. Prueba a buscar en http://scholar.google.com por referencias
% y vers que lo puedes hacer de manera sencilla.
% Ms informacin: 
% http://texblog.org/2014/04/22/using-google-scholar-to-download-bibtex-citations/

\end{document}
